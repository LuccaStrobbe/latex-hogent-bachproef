%%=============================================================================
%% Conclusie
%%=============================================================================

\chapter{Conclusie}%
\label{ch:conclusie}

% TODO: Trek een duidelijke conclusie, in de vorm van een antwoord op de
% onderzoeksvra(a)g(en). Wat was jouw bijdrage aan het onderzoeksdomein en
% hoe biedt dit meerwaarde aan het vakgebied/doelgroep? 
% Reflecteer kritisch over het resultaat. In Engelse teksten wordt deze sectie
% ``Discussion'' genoemd. Had je deze uitkomst verwacht? Zijn er zaken die nog
% niet duidelijk zijn?
% Heeft het onderzoek geleid tot nieuwe vragen die uitnodigen tot verder 
%onderzoek?

De resultaten van dit onderzoek hebben kunnen aantonen dat een geautomatiseerde aanpak van de reconnaissance fase de efficiëntie aanzienlijk kan verhogen.
Waar tools als RustScan volledige scans kunnen voltooien in minder dan een minuut, gaven AutoRecon en Sn1per een uitgebreide evaluatie van resultaten, in een langere uitvoeringstijd (~79 tot 129 minuten).
Cyberscan en Nuclei waren efficiënt voor een relatief snelle verkenning en vulnerability detectie, waar Layzyrecon en ReconFTW beperkter waren door de offline omgeving.
Deze konden door hun manuele aanpak een diepere verkenning uitvoeren, weliswaar ten kostte van de uitvoeringstijd.
Belangrijk te melden is dat de tijd van de manuele aanpak (twee tot drie dagen, ~30 minuten met bash script) enkel maar de actieve verwerkingstijd omvat.
Het analyseren van de commando's en het onderzoeken naar volgcommando's vraagt tijd, wat de algemene tijd van de manuele methode aanzienlijk vergroot.
Bovendien is de tijd van de installatie van de tools, de gefaalde tools en commando's niet meegerekend in dit rapport.
Een combinatie van de tool RustScan, voor snelheid, en Sn1per, voor de diepgang bleek het meest effectief te zijn volgens het OSSTMM Framework.
Een kritische beoordeling van de resultaten bevestigt de verwachtingen dat dat automatisatie de uitvoeringstijd en consistentie verbetert, maar dat de manuele aanpak beter is voor diepgaande analyses. Al was het verschil meer aanwezig dan oorspronkelijk gedacht.
De tijd van het analyseren en bedenken van een vervolgcommando speelde een grotere factor dan initieel gedacht.
Deze studie heeft ook nieuwe vragen naarboven gehaald zoals, hoe automatische tools werken op dynamische netwerken, of hybride aanpakken verder kunnen worden geoptimaliseerd en hoe mislukte commando's kunnen worden onderzocht om de workflow te verbeteren?
Deze vragen kunnen in een toekomstig onderzoeken worden beantwoord. 
Daarnaast vormt de toepassing van machine learning voor intelligente toolselectie een veelbelovende onderzoeksrichting, gezien het potentieel om de automatische reconnaisansse verder te optimaliseren.

% Bovendien verdient de toepassing van machine-learning voor Intelligente tool selectie is een intersante een nader onderzoek.
% Daarnaast vormt de toepassing van machine-learning voor Intelligente tool selectie is een veelbelovende onderzoeksrichting.

% Een bijzonder interessant pad naar de implementatie van machine learning-algoritmen voor geautomatiseerde selectie van security tools blijft open staan.
%  betreft de vraag van machine-learning voor Intelligente tool selectie, wat naar deverwachtingen de efficiëntie aanzienlijk zou verhogen.
% betreft de implementatie van machine learning-algoritmen voor geautomatiseerde selectie van security tools


% Daarnaast vormt de toepassing van machine-learning voor Intelligente tool selectie is een intersante onderzoeksrichting.