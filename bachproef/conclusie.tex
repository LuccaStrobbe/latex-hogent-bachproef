%%=============================================================================
%% Conclusie
%%=============================================================================

\chapter{Conclusie}%
\label{ch:conclusie}

% TODO: Trek een duidelijke conclusie, in de vorm van een antwoord op de
% onderzoeksvra(a)g(en). Wat was jouw bijdrage aan het onderzoeksdomein en
% hoe biedt dit meerwaarde aan het vakgebied/doelgroep? 
% Reflecteer kritisch over het resultaat. In Engelse teksten wordt deze sectie
% ``Discussion'' genoemd. Had je deze uitkomst verwacht? Zijn er zaken die nog
% niet duidelijk zijn?
% Heeft het onderzoek geleid tot nieuwe vragen die uitnodigen tot verder 
%onderzoek?

De resultaten van dit onderzoek heeft kunnen aantonen dat de automatisch aanpak van de reconnaissance fase de efficiëntie aanzienlijk kunnen verhogen.
Waar tools als RustScan volledige scans kunnen voltooien in minder dan een minuut terwijl AutoRecon en Sn1per een uitgebreide evaluatie van resultaten kon geven, zijnde het met langer uitvoeringstijd (~79-129min).
Cyberscan en Nuclei waren efficiënt voor een relatieve snelle verkenning en vulnerability detectie, terwijl Layzyrecon en ReconFTW beperkter waren door de offline omgeving.
De manuele aanpak een veel diepere verkenning kon nemen komt dit ten kostte van de uitvoeringstijd.
Belangerijk is te melden dat de tijd van de manuele aanpak (2-3 dagen, ~30 min met Bash script) enkel maar de actieve verwerkingstijd omvat.
Het analyseren van de commando's en het onderzoeken naar volgcomandos vraagt veel tijd, wat de algemene tijd van de manuele methode aanzienlijk vergroot.
Bovendien is de tijd van de installatie van de tools en de gefaalde en tools en commando's niet meegerekend in dit rapport.
Een combinatie van de tool RustScan voor snelheid en Sn1per voor de diepgang bleek het meest effectieve te zijn volgens het OSSTMM Framework.
Een kritische beoordeling van de resultaten bevestigen de verwachtingen dat dat automatisatie de uitvoeringstijd en consistentie verbeterd, maar dat de manuele aanpak veel beter ging zijn voor diepgaande analyse was meer aanwezig of oorspronkelijk gedacht.
De Tijd bij het analyseren en het bedenken van een vervolgcommando speelde een grotere factor dan oorspronkelijk gedacht.
Deze studie heeft ook nieuwe vragen naar boven gehaald zoals, hoe automatische tools werken op dynamische netwerken, of hybride aanpakken verder kan worden geoptimaliseerd, hoe mislukte commando's kan worden onderzocht om de workflow te verbeteren.
Deze vragen kunnen in een toekomstige onderzoeken dit ontdekken, ook de toepassing van machine learning voor Intelligente tool selectie.