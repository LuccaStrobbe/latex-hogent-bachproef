%%=============================================================================
%% Voorwoord
%%=============================================================================

\chapter*{\IfLanguageName{dutch}{Woord vooraf}{Preface}}%
\label{ch:voorwoord}

%% TODO:
%% Het voorwoord is het enige deel van de bachelorproef waar je vanuit je
%% eigen standpunt (``ik-vorm'') mag schrijven. Je kan hier bv. motiveren
%% waarom jij het onderwerp wil bespreken.
%% Vergeet ook niet te bedanken wie je geholpen/gesteund/... heeft

Het uitwerken van mijn bachelorproef was voor mij een belangrijke stap in zowel mijn academische als professionele ontwikkeling.
Gedreven door mijn passie voor cybersecurity, een vak dat slechts een klein deel van mijn opleiding omvangt, maar desondanks mijn volledige interesse heeft.
Deze studie behandelt een vergelijking tussen de automatisatie en manuele reconnaissance binnen een penetratietest. \\

Ik zou ook graag enkele personen willen bedanken want zonder hun zou mijn bachelorproef niet tot stand zijn gekomen.
Met name zou ik graag mijn promotor Andy Van Maele en co-promotor Stef Geeurickx willen bedanken voor hun waardevolle feedback, ondersteuning en expertise.
Tot slot zou ik graag mijn familie en vrienden willen bedanken voor hun steun en motivatie gedurende het hele traject. \\

Ik hoop dat deze studie niet enkel een bijdrage levert aan academische kennis, maar ook anderen inspiratie geeft en aanspoort om verdere toepassingen en onderzoek uit te voeren. 
In het snel evaluerende domein van ethisch hacken, waar beveiliging belangrijker is dan ooit, blijf voortdurend onderzoek essentieel.

% het uitwerken van mijn Bachelor Proef was zowel uitagend als belonend, het zetten van deze belangerijke mijlpaal zowel in mijn academische en profecionele carrière.
% Deze bachelor proef ondekt de automatisatie binnen de reconnaisansse fase van penetratietesten 