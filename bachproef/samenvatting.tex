%%=============================================================================
%% Samenvatting
%%=============================================================================

% TODO: De "abstract" of samenvatting is een kernachtige (~ 1 blz. voor een
% thesis) synthese van het document.
%
% Een goede abstract biedt een kernachtig antwoord op volgende vragen:
%
% 1. Waarover gaat de bachelorproef?
% 2. Waarom heb je er over geschreven?
% 3. Hoe heb je het onderzoek uitgevoerd?
% 4. Wat waren de resultaten? Wat blijkt uit je onderzoek?
% 5. Wat betekenen je resultaten? Wat is de relevantie voor het werkveld?
%
% Daarom bestaat een abstract uit volgende componenten:
%
% - inleiding + kaderen thema
% - probleemstelling
% - (centrale) onderzoeksvraag
% - onderzoeksdoelstelling
% - methodologie
% - resultaten (beperk tot de belangrijkste, relevant voor de onderzoeksvraag)
% - conclusies, aanbevelingen, beperkingen
%
% LET OP! Een samenvatting is GEEN voorwoord!

%%---------- Nederlandse samenvatting -----------------------------------------
%
% TODO: Als je je bachelorproef in het Engels schrijft, moet je eerst een
% Nederlandse samenvatting invoegen. Haal daarvoor onderstaande code uit
% commentaar.
% Wie zijn bachelorproef in het Nederlands schrijft, kan dit negeren, de inhoud
% wordt niet in het document ingevoegd.

\IfLanguageName{english}{%
\selectlanguage{dutch}
\chapter*{Samenvatting}
% \lipsum[1-4]
\selectlanguage{english}
}{}

%%---------- Samenvatting -----------------------------------------------------
% De samenvatting in de hoofdtaal van het document

\chapter*{\IfLanguageName{dutch}{Samenvatting}{Abstract}}

Penetratietesten zijn cruciaal voor het identificeren van kwetsbaarheden binnen systemen en netwerken, waar de reconnaissancefase de eerste en ook de belangrijkste rol speelt in het verzamelen van informatie.
Helaas is deze eerste stap tijdsintensief en gevoelig voor fouten.\\

Deze bachelorproef onderzoekt hoe automatisatie binnen de reconnaisanssefase de efficiëntie in pentesten verbetert en het identificeren van de meest effectieve tool.
De vergelijkende studie werd uitgevoerd in een gecontroleerde virtuele omgeving gebruik makende van een kali en metasploitable2 virtuele machine, de manuele reconnaissance (54 commando's over vijf OSSTMM gebaseerde categorieën) werdt vergeleken met geautomatiseerde tools als AutoRecon, Cyberscan, Sn1per, RustScan, Nuclei, LazyRecon en ReconFTW.
De resultaten werden geëvalueerd op basis van betrouwbaarheid, uitvoeringstijd, CPU-gebruik, bruikbaarheid en OSSTMM Risk Assessment Values (RAV) scores.
De manuele commando's werden ook in een bash script verwerkt hun de consistentie te testen. De automatisatie tools werden dus gecontroleerd op consistentie en op de mogelijkheid om de workflow efficiënter te maken.\\


Deze resultaten toonden aan dat automatisatie op vlak van uitvoeringstijd (met name RustScan) en een gedetaileerde evalualtie(Autorecon en Sn1per)
% Deze resultaten toonden aan dat automatisatie, voornamelijk RustScan voor snelheid en AutoRecon en Sn1per voor een uitgebreide evaluatie, 
drastisch de uitvoeringstijd en de consistentie verbetert, hoewel de manuele aanpak een dieper inzicht kan geven.
De bevindingen geven aan dat het combineren van tools zoals RustScan en Sn1per de efficiëntie verbetert, wat meer nuttige informatie kan opbrengen voor een pentester om de reconnaissancefase binnen het OSSTMM framework te verbeteren.
