%==============================================================================
% Sjabloon onderzoeksvoorstel bachproef
%==============================================================================
% Gebaseerd op document class `hogent-article'
% zie <https://github.com/HoGentTIN/latex-hogent-article>

% Voor een voorstel in het Engels: voeg de documentclass-optie [english] toe.
% Let op: kan enkel na toestemming van de bachelorproefcoördinator!
\documentclass{hogent-article}

% Invoegen bibliografiebestand
\addbibresource{voorstel.bib}

% Informatie over de opleiding, het vak en soort opdracht
\studyprogramme{Professionele bachelor toegepaste informatica}
\course{Bachelorproef}
\assignmenttype{Onderzoeksvoorstel}
% Voor een voorstel in het Engels, haal de volgende 3 regels uit commentaar
% \studyprogramme{Bachelor of applied information technology}
% \course{Bachelor thesis}
% \assignmenttype{Research proposal}

\academicyear{2024-2025} % TODO: pas het academiejaar aan

% TODO: Werktitel
\title{Vul hier de voorgestelde titel van je onderzoek in}

% TODO: Studentnaam en emailadres invullen
\author{Lucca Strobbe}
\email{Lucca.Strobbe@student.hogent.be}

% TODO: Medestudent
% Gaat het om een bachelorproef in samenwerking met een student in een andere
% opleiding? Geef dan de naam en emailadres hier
% \author{Yasmine Alaoui (naam opleiding)}
% \email{yasmine.alaoui@student.hogent.be}

% TODO: Geef de co-promotor op
\supervisor[Co-promotor]{S. Beekman (Synalco, \href{mailto:sigrid.beekman@synalco.be}{sigrid.beekman@synalco.be})}

% Binnen welke specialisatierichting uit 3TI situeert dit onderzoek zich?
% Kies uit deze lijst:
%
% - Mobile \& Enterprise development
% - AI \& Data Engineering
% - Functional \& Business Analysis
% - System \& Network Administrator
% - Mainframe Expert
% - Als het onderzoek niet past binnen een van deze domeinen specifieer je deze
%   zelf
%
\specialisation{System \& Network Administrator}
\keywords{Scheme, World Wide Web, $\lambda$-calculus} %ToDo

\begin{document}

\begin{abstract}
  Phishing blijft een van de meest voorkomende aanvalstechnieken, en traditionele phishing-awareness tests bieden vaak beperkte inzichten in het gedrag van medewerkers bij het herkennen van dergelijke aanvallen. Dit onderzoek richt zich op het ontwikkelen van een geavanceerd phishing-awareness model dat niet alleen test of medewerkers phishing-berichten herkennen, maar ook hun reactiepatronen analyseert. De centrale onderzoeksvraag is: Hoe kan een model voor phishing-awareness worden ontwikkeld dat inzicht biedt in de gedragingen van medewerkers en gerichte training optimaliseert? Het doel van dit onderzoek is het ontwikkelen van een proof-of-concept applicatie die phishing simulaties uitvoert, gedragsgegevens verzamelt en aanbevelingen doet voor gepersonaliseerde trainingen. Dit model voegt waarde toe aan de organisatie door trainingen te optimaliseren, wat op zijn beurt het risico op succesvolle phishing-aanvallen vermindert.
  %Hier schrijf je de samenvatting van je voorstel, als een doorlopende tekst van één paragraaf. Let op: dit is geen inleiding, maar een samenvattende tekst van heel je voorstel met inleiding (voorstelling, kaderen thema), probleemstelling en centrale onderzoeksvraag, onderzoeksdoelstelling (wat zie je als het concrete resultaat van je bachelorproef?), voorgestelde methodologie, verwachte resultaten en meerwaarde van dit onderzoek (wat heeft de doelgroep aan het resultaat?).
\end{abstract}

\tableofcontents

% De hoofdtekst van het voorstel zit in een apart bestand, zodat het makkelijk
% kan opgenomen worden in de bijlagen van de bachelorproef zelf.
%---------- Inleiding ---------------------------------------------------------

% TODO: Is dit voorstel gebaseerd op een paper van Research Methods die je
% vorig jaar hebt ingediend? Heb je daarbij eventueel samengewerkt met een
% andere student?
% Zo ja, haal dan de tekst hieronder uit commentaar en pas aan.

%\paragraph{Opmerking}

% Dit voorstel is gebaseerd op het onderzoeksvoorstel dat werd geschreven in het
% kader van het vak Research Methods dat ik (vorig/dit) academiejaar heb
% uitgewerkt (met medesturent VOORNAAM NAAM als mede-auteur).
% 




\section{Inleiding}%
\label{sec:inleiding}

Penetratietests(pentests) hebben een cruciale rol bij cybersecurity, waarbij cybersecurity en IT-professionals proberen in te breken in het intern netwerk 
van een organisatie of bedrijf om toegang te krijgen tot het systeem, dit om te voorkomen dat er black hat hackers toegang kunnen krijgen tot hun systeem. 
De eerste fase binnen dit proces in de reconnaissance-fase, waar er zo veel mogelijk data wordt verzameld over het doelwit. Dit is één van de belangrijkste fases, 
aangezien deze fase de basis legt voor rest van de pentest. Als deze fase handmatig wordt uitgevoerd zal deze stap te veel tijd kosten. 
%Met efficiëntie als doel zoekt dit onderzoek naar geautomatiseerde tools om het handmatige te vervangen.
het doel is om efficiëntie te bereiken door het vinden van geautomatiseerde tools om het handmatige proces te vervangen.
Het onderzoek richt zich op het vergelijken van de beschikbare tools en technieken om er de meest efficiënte tool of framework uit te halen, weliswaar zonder kwaliteitsverlies van de verzamelde informatie.
De belangerijkste doelgroep waar mijn studie op richt zijn IT’ers die werken met pentests, zoals cybersecurity- en IT-professionals en bedrijven die regelmatig hun systemen testen op kwetsbaarheden. 
Deze doelgroep is op zoek naar een betrouwbare tool dat efficiënt is en een snelle werking heeft. 
%Deze literatuuronderzoek heeft betrekking op o.a.
Om dit te analyseren zal ik een literatuuronderzoek doen naar enkele bestaande tools waaronder 
AutoRecon, Recon-ng en Shodan, aangevuld met testen op pentest scenario’s om deze werking te analyseren en hun efficiëntie te vergelijken. 
Het eindresultaat van dit onderzoek zal bestaan uit aanbevelingen voor de intigratie en werking van deze tools.


% Waarover zal je bachelorproef gaan? Introduceer het thema en zorg dat volgende zaken zeker duidelijk aanwezig zijn:

% \begin{itemize}
%   \item kaderen thema
%   \item de doelgroep
%   \item de probleemstelling en (centrale) onderzoeksvraag
%   \item de onderzoeksdoelstelling
% \end{itemize}

% Denk er aan: een typische bachelorproef is \textit{toegepast onderzoek}, wat betekent dat je start vanuit een concrete probleemsituatie in bedrijfscontext, een \textbf{casus}. Het is belangrijk om je onderwerp goed af te bakenen: je gaat voor die \textit{ene specifieke probleemsituatie} op zoek naar een goede oplossing, op basis van de huidige kennis in het vakgebied.

% De doelgroep moet ook concreet en duidelijk zijn, dus geen algemene of vaag gedefinieerde groepen zoals \emph{bedrijven}, \emph{developers}, \emph{Vlamingen}, enz. Je richt je in elk geval op it-professionals, een bachelorproef is geen populariserende tekst. Eén specifiek bedrijf (die te maken hebben met een concrete probleemsituatie) is dus beter dan \emph{bedrijven} in het algemeen.

% Formuleer duidelijk de onderzoeksvraag! De begeleiders lezen nog steeds te veel voorstellen waarin we geen onderzoeksvraag terugvinden.

% Schrijf ook iets over de doelstelling. Wat zie je als het concrete eindresultaat van je onderzoek, naast de uitgeschreven scriptie? Is het een proof-of-concept, een rapport met aanbevelingen, \ldots Met welk eindresultaat kan je je bachelorproef als een succes beschouwen?



%---------- Stand van zaken ---------------------------------------------------

\section{Literatuurstudie}%
\label{sec:literatuurstudie}

\subsection{Reconnaissance in penetratietests}

De reconnaissance-fase, of de verkenningsfase, is een fase binnen pentests die een cruciale rol heeft, maar vaak ook de meest tijdrovende stap, 
het kan weken tot maanden duren, afhankelijk van de complexiteit van het target en de technieken die worden gebruikt~\autocite{Shah}. 

Het doel van de reconnaissance-fase binnen pentests is het creëren van een volledig beeld om de 
kwetsbaarheden van het doelwit bloot te leggen. Dit process legt de basis voor de
volgende stappen in de pentest \autocite{Kothia}.

Tijdens deze fase maken testers of aanvallers gebruik van actieve en passieve technieken om zoveel
mogelijk informatie verzamelen. De passieve methode houdt in dat er informatie over het doelwit wordt verzameld zonder dat er direct contact mee wordt gemaakt, 
de actieve methode is wel direct verbonden met het doelwit, bijvoorbeeld door gebruik te maken van tools zoals Nmap, om open poorten te detecteren~\autocite{Shah}.


\subsection{Automatisering}

~\textcite{Hoang} heeft een geautomatiseerde aanpak voor pentesten waarin hij deep reinforcement 
learning (RL) gebruikt. Zijn model is gebaseerd op het A3C-algoritme. Dit algoritme leert zichzelf de geschikte acties aan, dit kan 
gaan over het kiezen van de payloads en het benutten van de juiste kwetsbaarheden. Zijn methode is gericht op drie functies: informatie verzamelen, 
exploitatie en rapportage. Hoang toont daarmee aan dat het gebruik van deze aanpak niet enkel de prestatie verbetert, maar ook dat het systeem de resultaten 
kan opslaan om die toe te passen op nieuwe situaties.~\autocite{Hoang}.


Volgens ~\textcite{Kothia} kan de eerste fase binnen pentesting in reconnaissance-fase veel 
efficiënter gebeuren met behulp van automatisatie. Er zijn veel open source tools beschikbaar, maar helaas vraagt het gebruik van deze 
tools nog veel handmatige inspanning om ze te integreren. Kothia heeft een geautomatiseerde aanpak gecreëerd om deze tools
efficiënter te integreren. De resultaten van de studie toonde aan dat deze implementatie, tijd 
bespaarde en zorgde voor een betere en nauwkeurigere uitvoering~\autocite{Kothia}.


\newline \subsection{Tools en Frameworks}
\newline Om pentests te automatiseren, zijn er verschillende frameworks en tools beschikbaar om 
efficiënt en snel informatie te verzamelen. De meeste van deze tools zijn open source, waaronder 
AutoRecon, Recon-ng, Shodan en Nmap. Deze tools zijn ontwikkeld om robust te zijn en het process van de Reconnaissance-fase te versnellen ~\autocite{Shebli}.


\subsection{Vergelijkingen}

Automatisering kan de efficiëntie en snelheid van de reconnaisance-fase verhogen. Er zijn vele tools die snel een netwerk kunnen scannen, 
er de kwetsbaarheden uit kunnen halen en de kwetsbaarheden kunnen detecteren. Door automatisatie kunnen 
organisaties sneller en kosten-effectiever tests uitvoeren. Deze automatisatie kan echter wel leiden tot een kwaliteitsverlies, 
d.w.z. dat mogelijks het systeem niet alle kwetsbaarheden vindt. ~\autocite{peris}

Meer diepgaande scans worden vaak handmatig gedaan, deze zijn flexibeler dan een geautomatiseerde test. 
Met de handmatige aanpak kun je makkelijker complexe kwetsbaarheden terugvinden die niet makkelijk te detecteren vallen bij 
een geautomatiseerde tool.~\autocite{techtarget} 

Vaak wordt er een keuze gemaakt tussen een handmatige of een geautomatiseerde penettest, deze keuze hangt vaak af van de specifieke  
vereisten van de organisatie. Echter wordt er ook niet gekozen om puur handmatig of geautomatiseerde pentest te maken, maar wordt
een hybride model vaker gebruikt. Dit is vaak de ideale oplossing waarbij automatisatie en menselijke ervaring elkaar kunnen 
aanvullen voor een efficiënte Reconnaissance-fase ~\autocite{techtarget}.

\subsection{Efficiëntie en uitdagingen}

% Momenteel zijn er nog enkele uitdagingen bij pentests, zoals ~\textcite{Fugkeaw} beschrijft, dat er nog te veel wordt vertrouwd op de experten van het vak bij het beoordelen van de al dan niet geautomatiseerde Reconnaissance-fase.
% Bij vele methoden wordt er vaak nog op gerekend dat er een menselijke input is bij het maken van een vulnerability assessment (VA) bij het analyseren en prioriteren. Dit verhoogt de kans op menselijke fouten en is tijdintensief, ook zijn de testen vaak 
% inconsistent door de individuele voorkeuren van de pentester volgens ~\textcite{Ghanem}.
% Binnen pentesting worden er nog vaak nieuwe technologieën ontdekt en gebruikt, één van deze nieuwe technologieën is het gebruik van kunstmatige intelligentie. Zoals Het Intelligent Automated Penetration Testing System (IAPTS), ontwikkeld door \textcite{Ghanem}, 
% hier wordt er gebruik gemaakt van reinforcement learning (RL) om testpatronen te leren en te optimaliseren. Als we deze technologie gebruiken in de reconnaissance-fase samen met het wiskundige model om beslissingen te maken; de Partially observed Markov decision process (POMDP) 
% kunnen we nauwkeurigere resultaten behalen. 
% Echter blijft het belangrijk dat er een juiste balans zit tussen automatisatie en menselijke controle. Menselijke testers hebben vaker meer creativiteit en intuïtie wat moeilijk is te vervangen door een geautomatiseerde tool. Daarom blijft het belangrijk om te blijven onderzoeken naar hybride modelen.

Momenteel zijn er nog enkele uitdagingen bij pentests, zoals ~\textcite{Fugkeaw} beschrijft, dat er nog te veel wordt vertrouwd op de experten van het vak bij het beoordelen van de al dan niet geautomatiseerde Reconnaissance-fase.Bij vele methoden wordt er vaak nog op gerekend dat er een menselijke input is bij het maken van een vulnerability assessment (VA) bij het analyseren en prioriteren. Dit verhoogt de kans op menselijke fouten en is tijdintensief, ook zijn de testen vaakinconsistent door de individuele voorkeuren van de pentester volgens ~\textcite{Ghanem}.
Binnen pentesting worden er nog vaak nieuwe technologieën ontdekt en gebruikt, één van deze nieuwe technologieën is het gebruik van kunstmatige intelligentie. Zoals Het Intelligent Automated Penetration Testing System (IAPTS), ontwikkeld door \textcite{Ghanem},hier wordt er gebruik gemaakt van reinforcement learning (RL) om testpatronen te leren en te optimaliseren. Als we deze technologie gebruiken in de reconnaissance-fase samen met het wiskundige model om beslissingen te maken; de Partially observed Markov decision process (POMDP) kunnen we nauwkeurigere resultaten behalen. Echter blijft het belangrijk dat er een juiste balans zit tussen automatisatie en menselijke controle. Menselijke testers hebben vaker meer creativiteit en intuïtie wat moeilijk is te vervangen door een geautomatiseerde tool. Daarom blijft het belangrijk om te blijven onderzoeken naar hybride modelen.




% Hier beschrijf je de \emph{state-of-the-art} rondom je gekozen onderzoeksdomein, d.w.z.\ een inleidende, doorlopende tekst over het onderzoeksdomein van je bachelorproef. Je steunt daarbij heel sterk op de professionele \emph{vakliteratuur}, en niet zozeer op populariserende teksten voor een breed publiek. Wat is de huidige stand van zaken in dit domein, en wat zijn nog eventuele open vragen (die misschien de aanleiding waren tot je onderzoeksvraag!)?

% Je mag de titel van deze sectie ook aanpassen (literatuurstudie, stand van zaken, enz.). Zijn er al gelijkaardige onderzoeken gevoerd? Wat concluderen ze? Wat is het verschil met jouw onderzoek?

%Verwijs bij elke introductie van een term of bewering over het domein naar de vakliteratuur, bijvoorbeeld~\autocite{Hykes2013}! Denk zeker goed na welke werken je refereert en waarom.

% Draag zorg voor correcte literatuurverwijzingen! Een bronvermelding hoort thuis \emph{binnen} de zin waar je je op die bron baseert, dus niet er buiten! Maak meteen een verwijzing als je gebruik maakt van een bron. Doe dit dus \emph{niet} aan het einde van een lange paragraaf. Baseer nooit teveel aansluitende tekst op eenzelfde bron.

% Als je informatie over bronnen verzamelt in JabRef, zorg er dan voor dat alle nodige info aanwezig is om de bron terug te vinden (zoals uitvoerig besproken in de lessen Research Methods).

% Voor literatuurverwijzingen zijn er twee belangrijke commando's:
% \autocite{KEY} => (Auteur, jaartal) Gebruik dit als de naam van de auteur
%   geen onderdeel is van de zin.
% \textcite{KEY} => Auteur (jaartal)  Gebruik dit als de auteursnaam wel een
%   functie heeft in de zin (bv. ``Uit onderzoek door Doll & Hill (1954) bleek
%   ...'')

% Je mag deze sectie nog verder onderverdelen in subsecties als dit de structuur van de tekst kan verduidelijken.

%---------- Methodologie ------------------------------------------------------
\section{Methodologie}%
\label{sec:methodologie}

Een combinatie van methoden zullen worden gebruikt om de onderzoeksvraag te beantwoorden, er zal onder andere gebruik
gemaakt worden van de literatuurstudie, onderzoeken en vergelijkende analyses. Hieronder zijn de stappen beschreven:

\subsection{Literatuurstudie}

De literatuurstudie zal de basis vormen voor dit onderzoek. Door een grondige analyse te maken van de reeds bestaande tools en studies,
die gebruikt worden in de reconnaissance-fase van pentests, kan er een overzicht gemaakt worden van de huidige situatie.
hierbij wordt er rekening gehouden met de vindingen van~\textcite{Shah,Kothia} die de voordelen en beperkingen aantonen binnen de reconnaissance-fase.
Ook zal er rekening gehouden worden met nieuwe technologieën zoals die van ~\textcite{Ghanem,Hoang} waar er gebruikt wordt gemaakt van
reinforcement learning (RL), Intelligent Automated Penetration Testing System (IAPTS) en POMDP-modellen. De implementatie van bestaande
tools en frameworks zoals AutoRecon en Recon-ng zal ook onderzocht worden~\autocite{Shebli}.

De resultaten van deze literatuurstudie zal worden gebruikt om een kader te schetsen voor de verdere ontwikkeling van dit onderzoek.

\subsection{Experimenteel onderzoek}

Om de effectiviteit van de automatisering te kunnen demonstreren in de reconnaissance-fase zal er een Een Proof of Concept (PoC) opgestart worden.
Dit omvat dat er een workflow zal worden geïmplementeerd met behulp van de automatisatie tools waaronder AutoRecon en Recon-ng. Deze
Workflow zal getest worden op een gesimuleerd netwerk waar er bekende kwetsbaarheden op staan om via deze data de snelheid,
nauwkeurigheid en consistentie te meten. Met behulp van deze data kunnen we nieuwe technologieën integreren om onze automatisatie te optimaliseren
zoals reinforcement learning (RL) om patronen te herkennen.
Aan de hand van de verzamelde informatie zal het onderzoek geëvalueerd worden aan de hand van nauwkeurigheid, tijdswinst en consistentie.


%----------------------------------------------------------------------------------------------------------
\subsection{Meetbare Criteria voor Analyse}

Om de resutaten van de analyse te beoordelen zal er gewerkt worden met de volgende meetbare criteria:
- De nodige tijd voor informatieverzameling.
- De Nauwkeurigheid van de ondekte gegevens.
- Het aantal gevonden unieke kwetsbaarheden.

Ook zullen er frameworks zoals OSTTMM en OWASp testing gebruikt worden om de valideit van het onderzoek nog te vergroten.
Deze richtlijnen en standaarden zal helpen om de resultaten van de PoC te vergelijken en te beoordelen.

Door deze criteria is het makkelijker om objectief te bepalen welke aanpak het meeste effect en efficiënt is.

%----------------------------------------------------------------------------------------------------------


\subsection{Vergelijkende studie}

Een vergelijkende analyse zal worden uitgevoerd om de toegevoegde waarde van de automatisering te bepalen. We zullen dit in 3 stappen kunnen uitvoeren:

\begin{itemize}
    \item Een handmatig uitgevoerde reconnaissance-fase, waarbij menselijke testers traditionele methoden toepassen.
    \item Een volledig geautomatiseerde aanpak met behulp van de PoC.
    \item Een hybride aanpak die handmatige en geautomatiseerde methoden combineert.
\end{itemize}

De resultaten van de analyse zullen we vergelijken op basis van enkele criteria zoals tijdsduur, nauwkeurigheid en de aantal gevonden kwetsbaarheden.

\subsection{Tools en hulpmiddelen}

In dit onderzoek zullen we gebruik maken van deze tools: 

\begin{itemize}
    \item Voor de verzamaling van informatie wordt er gebruik gemaakkt van Open-source frameworks waaronder AutoRecon, Nmap, Amass en Recon-ng.
    \item Python of andere talen voor het ontwerpen van de Poc
    \item voor de Simulatie zullen we gebruik maken van een lokaal virtueel gesimuleerd netwerk
\end{itemize}

De machine learning libraries (bijv. TensorFlow of PyTorch) zullen ook gebruikt worden voor de experimenten met reinforcement learning.

\subsection{Planning en deliverables}
Er zijn 4 fasen in het onderzoek:
\begin{enumerate}
    \item \textbf{Literatuurstudie} (3 weken): Relevante bronnen zullen worden geanalyseerd en de onderzoeksresultaten worden geïdentificeerd.
    \item \textbf{Proof of Concept} (5 weken): De geautomatiseerde workflow wordt ontwikkeld en geïmplementeerd.
    \item \textbf{Experimentele evaluatie} (4 weken): De PoC wordt getest in een gesimuleerde omgevingen.
    \item \textbf{Vergelijkende analyse en rapportage} (3 weken): De resultaten vergelijken en het schrijven van het eindrapport.
\end{enumerate}

De belangrijkste deliverables dat we willen bereiken is het automatiseren van de reconnaissance-fase op
basis van een onderzoeksrapport met aanbevelingen, en een Proof of Concept uitrollen waar we de voordelen
van automatisatie kunnen demonstreren.

% Hier beschrijf je hoe je van plan bent het onderzoek te voeren. Welke onderzoekstechniek ga je toepassen om elk van je onderzoeksvragen te beantwoorden? Gebruik je hiervoor literatuurstudie, interviews met belanghebbenden (bv.~voor requirements-analyse), experimenten, simulaties, vergelijkende studie, risico-analyse, PoC, \ldots?

% Valt je onderwerp onder één van de typische soorten bachelorproeven die besproken zijn in de lessen Research Methods (bv.\ vergelijkende studie of risico-analyse)? Zorg er dan ook voor dat we duidelijk de verschillende stappen terug vinden die we verwachten in dit soort onderzoek!

% Vermijd onderzoekstechnieken die geen objectieve, meetbare resultaten kunnen opleveren. Enquêtes, bijvoorbeeld, zijn voor een bachelorproef informatica meestal \textbf{niet geschikt}. De antwoorden zijn eerder meningen dan feiten en in de praktijk blijkt het ook bijzonder moeilijk om voldoende respondenten te vinden. Studenten die een enquête willen voeren, hebben meestal ook geen goede definitie van de populatie, waardoor ook niet kan aangetoond worden dat eventuele resultaten representatief zijn.

% Uit dit onderdeel moet duidelijk naar voor komen dat je bachelorproef ook technisch voldoen\-de diepgang zal bevatten. Het zou niet kloppen als een bachelorproef informatica ook door bv.\ een student marketing zou kunnen uitgevoerd worden.

% Je beschrijft ook al welke tools (hardware, software, diensten, \ldots) je denkt hiervoor te gebruiken of te ontwikkelen.

% Probeer ook een tijdschatting te maken. Hoe lang zal je met elke fase van je onderzoek bezig zijn en wat zijn de concrete \emph{deliverables} in elke fase?

%---------- Verwachte resultaten ----------------------------------------------
\section{Verwacht resultaat, conclusie}%
\label{sec:verwachte_resultaten}

% Hier beschrijf je welke resultaten je verwacht. Als je metingen en simulaties uitvoert, kan je hier al mock-ups maken van de grafieken samen met de verwachte conclusies. Benoem zeker al je assen en de onderdelen van de grafiek die je gaat gebruiken. Dit zorgt ervoor dat je concreet weet welk soort data je moet verzamelen en hoe je die moet meten.

% Wat heeft de doelgroep van je onderzoek aan het resultaat? Op welke manier zorgt jouw bachelorproef voor een meerwaarde?

% Hier beschrijf je wat je verwacht uit je onderzoek, met de motivatie waarom. Het is \textbf{niet} erg indien uit je onderzoek andere resultaten en conclusies vloeien dan dat je hier beschrijft: het is dan juist interessant om te onderzoeken waarom jouw hypothesen niet overeenkomen met de resultaten.

Ik verwacht dat ik uit mijn onderzoek een concreet inzicht krijg over hoe we de efficiëntie kunnen verbeteren met behulp van
automatisatie in de reconnaissance-fase van de pentest. Ik verwacht dat de implementatie van een hybride model het efficiëntste
zal zijn waar een er geavanceerde tool en framework zal gebruikt kunnen worden zoals AutoRecon, Shodan, Nmap en Recon-ng,
gecombineerd met menselijke expertise. Dit zal er voor zorgen dat er een snellere, betrouwbaardere en nauwkeurigere verzameling van informatie is.

Dit onderzoek zal bijdragen aan efficiëntere pentests, ook zal dit meer inzicht kunnen bieden hoe organisaties tijd en kosten
kunnen besparen zonder kwaliteitsverlies.

De verwachte uitkomsten omvatten:
\begin{itemize}
    \item Een gedetailleerde onderzoeksrapport van bestaande tools en frameworks om de pentests te automatiseren.
    \item Een Proof of Concept uitrollen waar we de voordelen van een hybride model kunnen demonstreren.
    \item Organisaties concrete aanbevelingen kunnen geven hoe hun pentestomgeving kan worden verbetert door automatisering.
\end{itemize}

Deze resultaten zullen afhankelijk zijn van de tests die uitgevoerd zullen worden in simulaties, dit geeft een goede basis om 
pentest-workflow te verbeteren in de praktijk.

\printbibliography[heading=bibintoc]

\end{document}