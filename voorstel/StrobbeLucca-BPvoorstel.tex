%==============================================================================
% Sjabloon onderzoeksvoorstel bachproef
%==============================================================================
% Gebaseerd op document class `hogent-article'
% zie <https://github.com/HoGentTIN/latex-hogent-article>

% Voor een voorstel in het Engels: voeg de documentclass-optie [english] toe.
% Let op: kan enkel na toestemming van de bachelorproefcoördinator!
\documentclass{hogent-article}

% Invoegen bibliografiebestand
\addbibresource{voorstel.bib}

% Informatie over de opleiding, het vak en soort opdracht
\studyprogramme{Professionele bachelor toegepaste informatica}
\course{Bachelorproef}
\assignmenttype{Onderzoeksvoorstel}
% Voor een voorstel in het Engels, haal de volgende 3 regels uit commentaar
% \studyprogramme{Bachelor of applied information technology}
% \course{Bachelor thesis}
% \assignmenttype{Research proposal}

\academicyear{2024-2025} % TODO: pas het academiejaar aan

% TODO: Werktitel
\title{Hoe kan de automatisering van de reconnaissance-fase de efficiëntie van penetratietests verbeteren, welke tools of frameworks kunnen hier het meest effectief bijdragen aan dit proces?}

% TODO: Studentnaam en emailadres invullen
\author{Lucca Strobbe}
\email{Lucca.Strobbe@student.hogent.be}

% TODO: Medestudent
% Gaat het om een bachelorproef in samenwerking met een student in een andere
% opleiding? Geef dan de naam en emailadres hier
% \author{Yasmine Alaoui (naam opleiding)}
% \email{yasmine.alaoui@student.hogent.be}

% TODO: Geef de co-promotor op
\supervisor[Co-promotor]{S. Beekman (Synalco, \href{mailto:sigrid.beekman@synalco.be}{sigrid.beekman@synalco.be})}

% Binnen welke specialisatierichting uit 3TI situeert dit onderzoek zich?
% Kies uit deze lijst:
%
% - Mobile \& Enterprise development
% - AI \& Data Engineering
% - Functional \& Business Analysis
% - System \& Network Administrator
% - Mainframe Expert
% - Als het onderzoek niet past binnen een van deze domeinen specifieer je deze
%   zelf
%
\specialisation{System \& Network Administrator}
\keywords{Automatisering ,Proof-of-Concept, Penetratietests, Cybersecurity} %ToDo

\begin{document} 

\begin{abstract}
  Door de groei van automatisatie tools in penetratietests zorgt voor een aanzienlijke verberging in efficiëntie van het verzamelen
  van informatie in de in reconnaissance-fase. Dit onderzoek zal onderzoeken hoe de automatisering van deze fase effect zal hebben
  op de efficiëntie van penetratietests om deze te verbeteren. De centrale onderzoeksvraag is : Hoe kan de automatisering van de 
  reconnaissance-fase de efficiëntie van penetratietests verbeteren, welke tools of frameworks kunnen hier het meest effectief 
  bijdragen aan dit proces? Het doel van dit onderzoek is om de impact van geautomatiseerde reconnaissance-tools en frameworks te 
  analyseren en te onderzoeken welke frameworks en tools de penetratietest kunnen verbeteren.  Dit onderzoek bevat ook een 
  literatuurstudie naar geautomatiseerde reconnaissance-tools zoals AutoRecon, Recon-ng en Shodan, aangevult met enkele praktische 
  experimeten in verschillende pentesting situaties om te testen hoe effectief en bruikbaar ze zijn. Dit onderzoek zal concrete 
  inzichten geven voor cybersecurity en IT-professionals om hun pentesting-workflow te optimaliseren, zonder dat er kwaliteitsverlies 
  voor kan doen.

  %Hier schrijf je de samenvatting van je voorstel, als een doorlopende tekst van één paragraaf. Let op: dit is geen inleiding, maar een samenvattende tekst van heel je voorstel met inleiding (voorstelling, kaderen thema), probleemstelling en centrale onderzoeksvraag, onderzoeksdoelstelling (wat zie je als het concrete resultaat van je bachelorproef?), voorgestelde methodologie, verwachte resultaten en meerwaarde van dit onderzoek (wat heeft de doelgroep aan het resultaat?).
\end{abstract}

\tableofcontents

% De hoofdtekst van het voorstel zit in een apart bestand, zodat het makkelijk
% kan opgenomen worden in de bijlagen van de bachelorproef zelf.
%---------- Inleiding ---------------------------------------------------------

% TODO: Is dit voorstel gebaseerd op een paper van Research Methods die je
% vorig jaar hebt ingediend? Heb je daarbij eventueel samengewerkt met een
% andere student?
% Zo ja, haal dan de tekst hieronder uit commentaar en pas aan.

%\paragraph{Opmerking}

% Dit voorstel is gebaseerd op het onderzoeksvoorstel dat werd geschreven in het
% kader van het vak Research Methods dat ik (vorig/dit) academiejaar heb
% uitgewerkt (met medesturent VOORNAAM NAAM als mede-auteur).
% 

\section{Inleiding}%
\label{sec:inleiding}

Penetratietests heeft een cruciale rol bij cybersecurity, waarbij cybersecurity en IT-professionals proberen binnen te breken in een organisatie of bedrijf om toegang te krijgen in het systeem, dit om te voorkomen dat er black hat hackers teoegan kunnen krijgen tot hun systeem. De eerste fase binnen  dit proces in de reconnaissance-fase, waarin er zo veel mogelijk data wordt verzamelt over het doelwit. Dit is een van de belangrijkste fase, doordat deze fase de basis legt voor rest van de pentest. Als deze fase handmatig wordt uitgevoerd zal deze stap daardoor nog vaak te veel tijd kosten. Hierdoor wordt er meer gekeken naar tools die deze fase kunnen automatiseren. Dit onderzoek zal kijken hoe de reconnaissance-fase kan worden verbetrd doot automatisering. Het doel van dit onderzoek is het vergelijken welke tools en technieken het meeste efficiënt kan zijn en waar er tijd wordt bespaart zonder kwaliteitsverlies.
De doelgroep van mijn onderzoek zijn IT’ers die werken met penetratietests, zoals cybersecurity en IT-professionals en bedrijven die regelmatig hun systemen testen op kwetsbaarheden. Deze doelgroep is op zoek naar een betrouwbare tool dat efficiënt en een snelle werking heeft. Om dit te onderzoeken zal ik een literatuuronderzoek doen naar enkele bestaande tools zoals AutoRecon, Recon-ng en Shodan. Aangevuld met testen op pentest scenario’s om deze werking te analyseren en de efficiëntie te vergelijken. Het eindresultaat van dit onderzoek zal bestaan uit aanbevelingen voor de intigratie en werking van tools voor de Penetratietests in de reconnaissance-fase.


% Waarover zal je bachelorproef gaan? Introduceer het thema en zorg dat volgende zaken zeker duidelijk aanwezig zijn:

% \begin{itemize}
%   \item kaderen thema
%   \item de doelgroep
%   \item de probleemstelling en (centrale) onderzoeksvraag
%   \item de onderzoeksdoelstelling
% \end{itemize}

% Denk er aan: een typische bachelorproef is \textit{toegepast onderzoek}, wat betekent dat je start vanuit een concrete probleemsituatie in bedrijfscontext, een \textbf{casus}. Het is belangrijk om je onderwerp goed af te bakenen: je gaat voor die \textit{ene specifieke probleemsituatie} op zoek naar een goede oplossing, op basis van de huidige kennis in het vakgebied.

% De doelgroep moet ook concreet en duidelijk zijn, dus geen algemene of vaag gedefinieerde groepen zoals \emph{bedrijven}, \emph{developers}, \emph{Vlamingen}, enz. Je richt je in elk geval op it-professionals, een bachelorproef is geen populariserende tekst. Eén specifiek bedrijf (die te maken hebben met een concrete probleemsituatie) is dus beter dan \emph{bedrijven} in het algemeen.

% Formuleer duidelijk de onderzoeksvraag! De begeleiders lezen nog steeds te veel voorstellen waarin we geen onderzoeksvraag terugvinden.

% Schrijf ook iets over de doelstelling. Wat zie je als het concrete eindresultaat van je onderzoek, naast de uitgeschreven scriptie? Is het een proof-of-concept, een rapport met aanbevelingen, \ldots Met welk eindresultaat kan je je bachelorproef als een succes beschouwen?



%---------- Stand van zaken ---------------------------------------------------

\section{Literatuurstudie}%
\label{sec:literatuurstudie}

\autocite{SabiEtAl2016}


\subsection{Inleiding tot reconnaissance in penetratietests}

Reconnaissance-fase, of de verkenningsfase, deze fase binnen penetratietests is een cruciale en vaak 
een tijdrovende stap. Deze fase is vaak de meest tijdrovende stap, dit kan weken of zelfs maanden 
duren, afhankelijk van de complexiteit van de target en de technieken die worden gebruikt~\autocite{Shah}. 

Het doel van de reconnaissance-fase binnen penetratietests is voor een volledig beeld te crieeren om de 
kwetsbaarheden van het doelwit open te leggen. Dit process is essentieel en legt de basis voor de
volgende stappen in penetratietest \autocite{Kothia}.

Tijdens deze fase maken testers of aanvallers gebruik van actieve en passieve technieken om zo veel
mogelijk informatie verzamelen van het doelwit. pasieven Methode is het verzalemlen van het doelwit 
zonder dat er dicect conct moet worden gemaakt met het doelwit, actieve methoden zijn dan wel direct
verbonden met het doelwit door te gaan scannen met tools zoals Nmap, rechtstreeks contact maken met 
het doelwit om open poorten te detecteren~\autocite{Shah}.


% Daarbij vereist de informatieverzamelingsfase een wetenschappelijke aanpak gecombineerd met
% creativiteit. Testers moeten elke mogelijke bron onderzoeken om een volledig beeld te krijgen
% van de aanvalsmogelijkheden~\autocite{doc3}.

% Het manueel zoeken naar informatie over een doelorganisatie is echter een tijdrovend en inefficiënt
% proces. Dit heeft geleid tot een sterke behoefte aan automatisering in de reconnaissance-fase. 
% Er zijn de afgelopen jaren diverse open-source tools ontwikkeld die specifieke informatie sneller
% en nauwkeuriger kunnen vinden~\autocite{doc3}. 
% Toch hebben deze tools beperkingen: ze zijn vaak niet interoperabel en vereisen handmatige 
% inspanningen om de resultaten te beheren en te analyseren~\autocite{doc3}.
% Dit benadrukt de noodzaak van verdere ontwikkeling van geautomatiseerde en geïntegreerde 
% frameworks die niet alleen efficiënt werken, maar ook schaalbaar zijn om grote volumes gegevens 
% te verwerken.







% Hier beschrijf je de \emph{state-of-the-art} rondom je gekozen onderzoeksdomein, d.w.z.\ een inleidende, doorlopende tekst over het onderzoeksdomein van je bachelorproef. Je steunt daarbij heel sterk op de professionele \emph{vakliteratuur}, en niet zozeer op populariserende teksten voor een breed publiek. Wat is de huidige stand van zaken in dit domein, en wat zijn nog eventuele open vragen (die misschien de aanleiding waren tot je onderzoeksvraag!)?

% Je mag de titel van deze sectie ook aanpassen (literatuurstudie, stand van zaken, enz.). Zijn er al gelijkaardige onderzoeken gevoerd? Wat concluderen ze? Wat is het verschil met jouw onderzoek?

%Verwijs bij elke introductie van een term of bewering over het domein naar de vakliteratuur, bijvoorbeeld~\autocite{Hykes2013}! Denk zeker goed na welke werken je refereert en waarom.

% Draag zorg voor correcte literatuurverwijzingen! Een bronvermelding hoort thuis \emph{binnen} de zin waar je je op die bron baseert, dus niet er buiten! Maak meteen een verwijzing als je gebruik maakt van een bron. Doe dit dus \emph{niet} aan het einde van een lange paragraaf. Baseer nooit teveel aansluitende tekst op eenzelfde bron.

% Als je informatie over bronnen verzamelt in JabRef, zorg er dan voor dat alle nodige info aanwezig is om de bron terug te vinden (zoals uitvoerig besproken in de lessen Research Methods).

% Voor literatuurverwijzingen zijn er twee belangrijke commando's:
% \autocite{KEY} => (Auteur, jaartal) Gebruik dit als de naam van de auteur
%   geen onderdeel is van de zin.
% \textcite{KEY} => Auteur (jaartal)  Gebruik dit als de auteursnaam wel een
%   functie heeft in de zin (bv. ``Uit onderzoek door Doll & Hill (1954) bleek
%   ...'')

% Je mag deze sectie nog verder onderverdelen in subsecties als dit de structuur van de tekst kan verduidelijken.

%---------- Methodologie ------------------------------------------------------
\section{Methodologie}%
\label{sec:methodologie}

% Hier beschrijf je hoe je van plan bent het onderzoek te voeren. Welke onderzoekstechniek ga je toepassen om elk van je onderzoeksvragen te beantwoorden? Gebruik je hiervoor literatuurstudie, interviews met belanghebbenden (bv.~voor requirements-analyse), experimenten, simulaties, vergelijkende studie, risico-analyse, PoC, \ldots?

% Valt je onderwerp onder één van de typische soorten bachelorproeven die besproken zijn in de lessen Research Methods (bv.\ vergelijkende studie of risico-analyse)? Zorg er dan ook voor dat we duidelijk de verschillende stappen terug vinden die we verwachten in dit soort onderzoek!

% Vermijd onderzoekstechnieken die geen objectieve, meetbare resultaten kunnen opleveren. Enquêtes, bijvoorbeeld, zijn voor een bachelorproef informatica meestal \textbf{niet geschikt}. De antwoorden zijn eerder meningen dan feiten en in de praktijk blijkt het ook bijzonder moeilijk om voldoende respondenten te vinden. Studenten die een enquête willen voeren, hebben meestal ook geen goede definitie van de populatie, waardoor ook niet kan aangetoond worden dat eventuele resultaten representatief zijn.

% Uit dit onderdeel moet duidelijk naar voor komen dat je bachelorproef ook technisch voldoen\-de diepgang zal bevatten. Het zou niet kloppen als een bachelorproef informatica ook door bv.\ een student marketing zou kunnen uitgevoerd worden.

% Je beschrijft ook al welke tools (hardware, software, diensten, \ldots) je denkt hiervoor te gebruiken of te ontwikkelen.

% Probeer ook een tijdschatting te maken. Hoe lang zal je met elke fase van je onderzoek bezig zijn en wat zijn de concrete \emph{deliverables} in elke fase?

%---------- Verwachte resultaten ----------------------------------------------
\section{Verwacht resultaat, conclusie}%
\label{sec:verwachte_resultaten}

% Hier beschrijf je welke resultaten je verwacht. Als je metingen en simulaties uitvoert, kan je hier al mock-ups maken van de grafieken samen met de verwachte conclusies. Benoem zeker al je assen en de onderdelen van de grafiek die je gaat gebruiken. Dit zorgt ervoor dat je concreet weet welk soort data je moet verzamelen en hoe je die moet meten.

% Wat heeft de doelgroep van je onderzoek aan het resultaat? Op welke manier zorgt jouw bachelorproef voor een meerwaarde?

% Hier beschrijf je wat je verwacht uit je onderzoek, met de motivatie waarom. Het is \textbf{niet} erg indien uit je onderzoek andere resultaten en conclusies vloeien dan dat je hier beschrijft: het is dan juist interessant om te onderzoeken waarom jouw hypothesen niet overeenkomen met de resultaten.



\printbibliography[heading=bibintoc]

\end{document}