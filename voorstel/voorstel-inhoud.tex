%---------- Inleiding ---------------------------------------------------------

% TODO: Is dit voorstel gebaseerd op een paper van Research Methods die je
% vorig jaar hebt ingediend? Heb je daarbij eventueel samengewerkt met een
% andere student?
% Zo ja, haal dan de tekst hieronder uit commentaar en pas aan.

%\paragraph{Opmerking}

% Dit voorstel is gebaseerd op het onderzoeksvoorstel dat werd geschreven in het
% kader van het vak Research Methods dat ik (vorig/dit) academiejaar heb
% uitgewerkt (met medesturent VOORNAAM NAAM als mede-auteur).
% 

\section{Inleiding}%
\label{sec:inleiding}

Penetratietests heeft een cruciale rol bij cybersecurity, waarbij cybersecurity en IT-professionals proberen binnen te breken in een organisatie of bedrijf om toegang te krijgen in het systeem, dit om te voorkomen dat er black hat hackers teoegan kunnen krijgen tot hun systeem. De eerste fase binnen  dit proces in de reconnaissance-fase, waarin er zo veel mogelijk data wordt verzamelt over het doelwit. Dit is een van de belangrijkste fase, doordat deze fase de basis legt voor rest van de pentest. Als deze fase handmatig wordt uitgevoerd zal deze stap daardoor nog vaak te veel tijd kosten. Hierdoor wordt er meer gekeken naar tools die deze fase kunnen automatiseren. Dit onderzoek zal kijken hoe de reconnaissance-fase kan worden verbetrd doot automatisering. Het doel van dit onderzoek is het vergelijken welke tools en technieken het meeste efficiënt kan zijn en waar er tijd wordt bespaart zonder kwaliteitsverlies.
De doelgroep van mijn onderzoek zijn IT’ers die werken met penetratietests, zoals cybersecurity en IT-professionals en bedrijven die regelmatig hun systemen testen op kwetsbaarheden. Deze doelgroep is op zoek naar een betrouwbare tool dat efficiënt en een snelle werking heeft. Om dit te onderzoeken zal ik een literatuuronderzoek doen naar enkele bestaande tools zoals AutoRecon, Recon-ng en Shodan. Aangevuld met testen op pentest scenario’s om deze werking te analyseren en de efficiëntie te vergelijken. Het eindresultaat van dit onderzoek zal bestaan uit aanbevelingen voor de intigratie en werking van tools voor de Penetratietests in de reconnaissance-fase.


% Waarover zal je bachelorproef gaan? Introduceer het thema en zorg dat volgende zaken zeker duidelijk aanwezig zijn:

% \begin{itemize}
%   \item kaderen thema
%   \item de doelgroep
%   \item de probleemstelling en (centrale) onderzoeksvraag
%   \item de onderzoeksdoelstelling
% \end{itemize}

% Denk er aan: een typische bachelorproef is \textit{toegepast onderzoek}, wat betekent dat je start vanuit een concrete probleemsituatie in bedrijfscontext, een \textbf{casus}. Het is belangrijk om je onderwerp goed af te bakenen: je gaat voor die \textit{ene specifieke probleemsituatie} op zoek naar een goede oplossing, op basis van de huidige kennis in het vakgebied.

% De doelgroep moet ook concreet en duidelijk zijn, dus geen algemene of vaag gedefinieerde groepen zoals \emph{bedrijven}, \emph{developers}, \emph{Vlamingen}, enz. Je richt je in elk geval op it-professionals, een bachelorproef is geen populariserende tekst. Eén specifiek bedrijf (die te maken hebben met een concrete probleemsituatie) is dus beter dan \emph{bedrijven} in het algemeen.

% Formuleer duidelijk de onderzoeksvraag! De begeleiders lezen nog steeds te veel voorstellen waarin we geen onderzoeksvraag terugvinden.

% Schrijf ook iets over de doelstelling. Wat zie je als het concrete eindresultaat van je onderzoek, naast de uitgeschreven scriptie? Is het een proof-of-concept, een rapport met aanbevelingen, \ldots Met welk eindresultaat kan je je bachelorproef als een succes beschouwen?



%---------- Stand van zaken ---------------------------------------------------

\section{Literatuurstudie}%
\label{sec:literatuurstudie}

\subsection{reconnaissance in penetratietests}

Reconnaissance-fase, of de verkenningsfase, deze fase binnen penetratietests is een cruciale en vaak 
een tijdrovende stap. Deze fase is vaak de meest tijdrovende stap, dit kan weken of zelfs maanden 
duren, afhankelijk van de complexiteit van de target en de technieken die worden gebruikt~\autocite{Shah}. 

Het doel van de reconnaissance-fase binnen penetratietests is voor een volledig beeld te crieeren om de 
kwetsbaarheden van het doelwit open te leggen. Dit process is essentieel en legt de basis voor de
volgende stappen in penetratietest \autocite{Kothia}.

Tijdens deze fase maken testers of aanvallers gebruik van actieve en passieve technieken om zo veel
mogelijk informatie verzamelen van het doelwit. pasieven Methode is het verzalemlen van het doelwit 
zonder dat er dicect conct moet worden gemaakt met het doelwit, actieve methoden zijn dan wel direct
verbonden met het doelwit door te gaan scannen met tools zoals Nmap, rechtstreeks contact maken met 
het doelwit om open poorten te detecteren~\autocite{Shah}.

\subsection{Automatisering}

~\textcite{Hoang} heeft een geautomatiseerde aanpak voor penetratietesten waar er gebruik wordt gemaakt van deep reinforcement 
learning (RL). Zijn model is gebaseerd op het A3C-algoritme, dit algorithme leert zichzelf de geschikte acties te herkennen, dit kan 
gaan over de payloads kiezen en de juiste kwetsbaarheden benutten. Zijn  methode is gericht op drie functies: informatieverzameling, 
exploitatie en rapportage. Hoang toont daarmee aan dat gebruik van deze aanpak niet enkel maat presatie verbetert maar ook de resultaten 
kan opslaan om hiermee later dit toe te passen op nieuwe situaties.~\autocite{Hoang}.

Volgens Kothia~\textcite{Kothia} kan de eerste fase binnen pentesting in reconnaissance-face waar er data wordt verzameld veel 
eficienter gebeuren met behulp van automatisatie. Er zijn vele open source tools beschikbaar maar helaas vraagt het gebruik van deze 
tools nog veel handmatige inspanning im deze tools te intigreren. Kothia heeft een geautomatiseerde aanpak gecrieert om deze tools
efficiënter te intigreren. De resultaten van de studie wees er op dat deze implementatie voor een snellere zorgde,tijd 
bespaarde en zorgde voor een betere en nauwkeurigere uitvoering in de reconnaissance-face\autocite{Kothia}.


\subsection{Tools en Frameworks}

Om penetratietests te automatiseren voor de reconnaissance-fase zijn er verschillende frameworks en tools die beschikbaar zijn om 
efficiënt en snel informatie kan verzalelen. Voornamelijk zijn de meeste van deze tools open source, enkele van deze tools zijn 
AutoRecon, Recon-ng, Shodan en Nmap. Deze tools zijn ontwikkeld om robust te zijn en het process Reconnaissance te versnellen~\autocite{Shebli}.


\subsection{Vergelijkingen}

Door Automatisering kan de efficiëntie en de snelheid van de reconnaissance fase aanzienlijk verhoogt worden. Er zijn vele tools om 
snel en efficiënt een netwerk te schannen en de kwedsbaarheden te kunnen detecteren, door de implementatie van automatisatie kunnen 
organisaties sneller en kosteneffectiever test kunnen uitvoeren. Echter, door de automatisatie kan het altijd
zijn dat deze tools niet alle kwetsbaarheden kan vinden en dit kan later voor problemen zorgen.~\autocite{peris}

Meer diepgaande scans worden vaak handmatig gedaan, deze zijn flexibel en de kan je meer doen als een geautomatiseerde test. 
Met de handmatige aanpak kun je makkelijker complexe kwetsbaarheden tegugvinden dat niet makkelijk te detecteren valt bij 
een geautomatiseerde tool.~\autocite{techtarget} Deze hadmatige test kunnen ook zeer tijdrovend zijn en kan van enkele 
weken tot enkele maaden duren afhankelijk van de complexiteit ~\autocite{Shah}.

Vaak wordt een keuze gemaake van tussen handmatig en een geautomatiseerde penetratietest, dit hangt hangt af van de specefieke 
vereisten van de oranisatie. Echter wordt er ook niet gekozen om puur handmatig of geautomatiseerde penetratietest te maken maar wordt 
een hybryde model vaker gebruikt, dit is vaak de ideale oplossing waarbij automatisatie em menselijke ervaring elkaar kunnen 
aanvullen voor een efficiënte Reconnaissance-fase~\autocite{techtarget}.

\subsection{Efficiëntie en uitdagingen}

Momenleel zijn er nog enkele uitdagingnen bij penetratietests,zoals \textcite{Fugkeaw} beschrijft dat er nog te veel wordt 
vertrouwd en afhankelijk zijn op de experten van het vak bij het beoorderelen van de al dan niet geautomatizeerde kwetsbaarheden. 
Bij vele methoden wordt er vaak nog op gerekend dat er een menselijke input bij het maken van een vulnerability assessment (VA) 
bij het analyseren en prioriteren. Dit verhoogt da kans op menselijke fouten en is meer tijdintensief, ook zijn de testen vaak 
inconsistent door de individuele voorkeren van de penetratietester volgens\textcite{Ghanem}.

Binnen pentesting worden er nog vaak nieuwe technologieen gebruikt, een van deze nieuwe technologieen is het gebruik van 
kunstmatige intelligentie> Zoals Het Intelligent Automated Penetration Testing System (IAPTS), ontwikkeld door \textcite{Ghanem},
hier wordt er gebruik gemaakt van reinforcement learning (RL) om testpartonen te leren en te optimaliseren. Als we deze technologie 
gebruiken in de reconnaissance-fase samen met de wiskundige model om beslisinge te maken de Partially observed Markov decision process (POMDP)
kunnen we naukeurigere resultaten behalen als menselijke testers.
Echter blijft het belangerijk dat er een juiste balans zit tussen automatisateie en menselijke controle. Menselijke testers hebben vaker 
meer creativiteit em intuïtie dat moeilijk is te vervangen door een geautomatiseerde tool. Daarom blijft het belangerijk om onderzoekt ted 
doen naar hybryde modelen.

Een andere veelbelovende ontwikkeling is het gebruik van kunstmatige intelligentie in pentesting. Het Intelligent Automated 
Penetration Testing System (IAPTS), ontwikkeld door \textcite{Ghanem}, maakt gebruik van reinforcement learning (RL) om 
teststrategieën te leren, te reproduceren en te optimaliseren. Door de reconnaissance-fase te modelleren als een partially observed 
Markov decision process (POMDP) kan IAPTS effectievere en consistentere resultaten leveren dan menselijke testers. Daarnaast biedt 
dit systeem de mogelijkheid om kennis en strategieën uit eerdere tests opnieuw te gebruiken, wat de efficiëntie verder verhoogt.






% Daarbij vereist de informatieverzamelingsfase een wetenschappelijke aanpak gecombineerd met
% creativiteit. Testers moeten elke mogelijke bron onderzoeken om een volledig beeld te krijgen
% van de aanvalsmogelijkheden~\autocite{doc3}.

% Het manueel zoeken naar informatie over een doelorganisatie is echter een tijdrovend en inefficiënt
% proces. Dit heeft geleid tot een sterke behoefte aan automatisering in de reconnaissance-fase. 
% Er zijn de afgelopen jaren diverse open-source tools ontwikkeld die specifieke informatie sneller
% en nauwkeuriger kunnen vinden~\autocite{doc3}. 
% Toch hebben deze tools beperkingen: ze zijn vaak niet interoperabel en vereisen handmatige 
% inspanningen om de resultaten te beheren en te analyseren~\autocite{doc3}.
% Dit benadrukt de noodzaak van verdere ontwikkeling van geautomatiseerde en geïntegreerde 
% frameworks die niet alleen efficiënt werken, maar ook schaalbaar zijn om grote volumes gegevens 
% te verwerken.







% Hier beschrijf je de \emph{state-of-the-art} rondom je gekozen onderzoeksdomein, d.w.z.\ een inleidende, doorlopende tekst over het onderzoeksdomein van je bachelorproef. Je steunt daarbij heel sterk op de professionele \emph{vakliteratuur}, en niet zozeer op populariserende teksten voor een breed publiek. Wat is de huidige stand van zaken in dit domein, en wat zijn nog eventuele open vragen (die misschien de aanleiding waren tot je onderzoeksvraag!)?

% Je mag de titel van deze sectie ook aanpassen (literatuurstudie, stand van zaken, enz.). Zijn er al gelijkaardige onderzoeken gevoerd? Wat concluderen ze? Wat is het verschil met jouw onderzoek?

%Verwijs bij elke introductie van een term of bewering over het domein naar de vakliteratuur, bijvoorbeeld~\autocite{Hykes2013}! Denk zeker goed na welke werken je refereert en waarom.

% Draag zorg voor correcte literatuurverwijzingen! Een bronvermelding hoort thuis \emph{binnen} de zin waar je je op die bron baseert, dus niet er buiten! Maak meteen een verwijzing als je gebruik maakt van een bron. Doe dit dus \emph{niet} aan het einde van een lange paragraaf. Baseer nooit teveel aansluitende tekst op eenzelfde bron.

% Als je informatie over bronnen verzamelt in JabRef, zorg er dan voor dat alle nodige info aanwezig is om de bron terug te vinden (zoals uitvoerig besproken in de lessen Research Methods).

% Voor literatuurverwijzingen zijn er twee belangrijke commando's:
% \autocite{KEY} => (Auteur, jaartal) Gebruik dit als de naam van de auteur
%   geen onderdeel is van de zin.
% \textcite{KEY} => Auteur (jaartal)  Gebruik dit als de auteursnaam wel een
%   functie heeft in de zin (bv. ``Uit onderzoek door Doll & Hill (1954) bleek
%   ...'')

% Je mag deze sectie nog verder onderverdelen in subsecties als dit de structuur van de tekst kan verduidelijken.

%---------- Methodologie ------------------------------------------------------
\section{Methodologie}%
\label{sec:methodologie}

% Hier beschrijf je hoe je van plan bent het onderzoek te voeren. Welke onderzoekstechniek ga je toepassen om elk van je onderzoeksvragen te beantwoorden? Gebruik je hiervoor literatuurstudie, interviews met belanghebbenden (bv.~voor requirements-analyse), experimenten, simulaties, vergelijkende studie, risico-analyse, PoC, \ldots?

% Valt je onderwerp onder één van de typische soorten bachelorproeven die besproken zijn in de lessen Research Methods (bv.\ vergelijkende studie of risico-analyse)? Zorg er dan ook voor dat we duidelijk de verschillende stappen terug vinden die we verwachten in dit soort onderzoek!

% Vermijd onderzoekstechnieken die geen objectieve, meetbare resultaten kunnen opleveren. Enquêtes, bijvoorbeeld, zijn voor een bachelorproef informatica meestal \textbf{niet geschikt}. De antwoorden zijn eerder meningen dan feiten en in de praktijk blijkt het ook bijzonder moeilijk om voldoende respondenten te vinden. Studenten die een enquête willen voeren, hebben meestal ook geen goede definitie van de populatie, waardoor ook niet kan aangetoond worden dat eventuele resultaten representatief zijn.

% Uit dit onderdeel moet duidelijk naar voor komen dat je bachelorproef ook technisch voldoen\-de diepgang zal bevatten. Het zou niet kloppen als een bachelorproef informatica ook door bv.\ een student marketing zou kunnen uitgevoerd worden.

% Je beschrijft ook al welke tools (hardware, software, diensten, \ldots) je denkt hiervoor te gebruiken of te ontwikkelen.

% Probeer ook een tijdschatting te maken. Hoe lang zal je met elke fase van je onderzoek bezig zijn en wat zijn de concrete \emph{deliverables} in elke fase?

%---------- Verwachte resultaten ----------------------------------------------
\section{Verwacht resultaat, conclusie}%
\label{sec:verwachte_resultaten}

% Hier beschrijf je welke resultaten je verwacht. Als je metingen en simulaties uitvoert, kan je hier al mock-ups maken van de grafieken samen met de verwachte conclusies. Benoem zeker al je assen en de onderdelen van de grafiek die je gaat gebruiken. Dit zorgt ervoor dat je concreet weet welk soort data je moet verzamelen en hoe je die moet meten.

% Wat heeft de doelgroep van je onderzoek aan het resultaat? Op welke manier zorgt jouw bachelorproef voor een meerwaarde?

% Hier beschrijf je wat je verwacht uit je onderzoek, met de motivatie waarom. Het is \textbf{niet} erg indien uit je onderzoek andere resultaten en conclusies vloeien dan dat je hier beschrijft: het is dan juist interessant om te onderzoeken waarom jouw hypothesen niet overeenkomen met de resultaten.

