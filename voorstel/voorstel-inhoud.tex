%---------- Inleiding ---------------------------------------------------------

% TODO: Is dit voorstel gebaseerd op een paper van Research Methods die je
% vorig jaar hebt ingediend? Heb je daarbij eventueel samengewerkt met een
% andere student?
% Zo ja, haal dan de tekst hieronder uit commentaar en pas aan.

%\paragraph{Opmerking}

% Dit voorstel is gebaseerd op het onderzoeksvoorstel dat werd geschreven in het
% kader van het vak Research Methods dat ik (vorig/dit) academiejaar heb
% uitgewerkt (met medesturent VOORNAAM NAAM als mede-auteur).
% 




\section{Inleiding}%
\label{sec:inleiding}

Penetratietests (pentests) hebben een cruciale rol bij cybersecurity, waarbij cybersecurity en IT-professionals proberen in te breken in het intern netwerk 
van een organisatie of bedrijf om toegang te krijgen tot het systeem, dit om te voorkomen dat er black hat hackers toegang kunnen krijgen tot hun systeem. 
De eerste fase binnen dit proces is de reconnaissance-fase, waar het verzamelen van zo veel mogelijk data over het doelwit centraal staat. Dit is één van de belangrijkste fases, 
aangezien deze fase de basis legt voor de rest van de pentest. Bij de handmatige uitgevoering van deze fase, zal de stap te veel tijd kosten. 
%Met efficiëntie als doel zoekt dit onderzoek naar geautomatiseerde tools om het handmatige te vervangen.
Het doel is efficiëntie te bereiken door het vinden van geautomatiseerde tools om het handmatige proces te vervangen.
Het onderzoek richt zich op het vergelijken van de beschikbare tools en technieken om er de meest efficiënte tool of het meest efficiënte framework uit te halen, weliswaar zonder kwaliteitsverlies van de verzamelde informatie.
De belangrijkste doelgroep waar de studie zich op richt zijn IT’ers die werken met pentests, zoals cybersecurity- en IT-professionals en bedrijven die regelmatig hun systemen testen op kwetsbaarheden. 
Deze doelgroep is op zoek naar een betrouwbare tool die efficiënt is en een snelle werking heeft. 
%Deze literatuuronderzoek heeft betrekking op o.a.
Om dit te analyseren zal er een literatuuronderzoek zijn naar enkele bestaande tools, waaronder 
AutoRecon, Recon-ng en Shodan, aangevuld met testen op pentest scenario’s om hun werking te analyseren en de efficiëntie te vergelijken. 
Het eindresultaat van dit onderzoek zal bestaan uit aanbevelingen voor de integratie en werking van deze tools.


% Waarover zal je bachelorproef gaan? Introduceer het thema en zorg dat volgende zaken zeker duidelijk aanwezig zijn:

% \begin{itemize}
%   \item kaderen thema
%   \item de doelgroep
%   \item de probleemstelling en (centrale) onderzoeksvraag
%   \item de onderzoeksdoelstelling
% \end{itemize}

% Denk er aan: een typische bachelorproef is \textit{toegepast onderzoek}, wat betekent dat je start vanuit een concrete probleemsituatie in bedrijfscontext, een \textbf{casus}. Het is belangrijk om je onderwerp goed af te bakenen: je gaat voor die \textit{ene specifieke probleemsituatie} op zoek naar een goede oplossing, op basis van de huidige kennis in het vakgebied.

% De doelgroep moet ook concreet en duidelijk zijn, dus geen algemene of vaag gedefinieerde groepen zoals \emph{bedrijven}, \emph{developers}, \emph{Vlamingen}, enz. Je richt je in elk geval op it-professionals, een bachelorproef is geen populariserende tekst. Eén specifiek bedrijf (die te maken hebben met een concrete probleemsituatie) is dus beter dan \emph{bedrijven} in het algemeen.

% Formuleer duidelijk de onderzoeksvraag! De begeleiders lezen nog steeds te veel voorstellen waarin we geen onderzoeksvraag terugvinden.

% Schrijf ook iets over de doelstelling. Wat zie je als het concrete eindresultaat van je onderzoek, naast de uitgeschreven scriptie? Is het een proof-of-concept, een rapport met aanbevelingen, \ldots Met welk eindresultaat kan je je bachelorproef als een succes beschouwen?



%---------- Stand van zaken ---------------------------------------------------

\section{Literatuurstudie}%
\label{sec:literatuurstudie}

\subsection{Reconnaissance in penetratietests}

De reconnaissance-fase, of de verkenningsfase, is een fase binnen pentests die een cruciale rol heeft, maar het is vaak ook de meest tijdrovende stap.
Het kan weken tot maanden duren, afhankelijk van de complexiteit van het target en de gebruikte technieken varieert de aanpak~\autocite{Shah}. 
Het doel van de reconnaissance-fase binnen pentests is het creëren van een volledig beeld om de 
kwetsbaarheden van het doelwit bloot te leggen. Dit proces legt de basis voor de
volgende stappen in de pentest \autocite{Kothia}. 
Tijdens deze fase maken testers of aanvallers gebruik van actieve en passieve technieken om zoveel
mogelijk informatie te verzamelen. De passieve methode houdt in informatie over het doelwit te verzamelen zonder er direct contact mee te maken, 
de actieve methode is wel direct verbonden met het doelwit, bijvoorbeeld door gebruik te maken van tools zoals Nmap, om open poorten te detecteren~\autocite{Shah}.

\subsection{Automatisering}

~\textcite{Hoang} heeft een geautomatiseerde aanpak voor pentesten waarin hij deep reinforcement 
learning (RL) gebruikt. Zijn model is gebaseerd op het A3C-algoritme. Dit algoritme leert zichzelf de geschikte acties aan, wat kan 
gaan over het kiezen van de payloads en het benutten van de juiste kwetsbaarheden. Zijn methode is gericht op drie functies: informatie verzamelen, 
exploitatie en rapportage. Hoang toont daarmee aan dat het gebruik van deze aanpak niet enkel de prestatie verbetert, maar ook dat het systeem de resultaten 
kan opslaan om die toe te passen op nieuwe situaties. ~\autocite{Hoang}
Volgens ~\textcite{Kothia} kan de eerste fase binnen pentesting in de reconnaissance-fase veel 
efficiënter gebeuren met behulp van automatisatie. Er zijn veel open source tools beschikbaar, maar helaas vraagt het gebruik van deze 
tools nog veel handmatige inspanning om ze te integreren. Kothia heeft een geautomatiseerde aanpak gecreëerd om deze tools
efficiënter te integreren. De resultaten van de studie toonden aan dat deze implementatie tijd 
bespaarde en zorgde voor een betere en nauwkeurigere uitvoering~\autocite{Kothia}.


\subsection{Tools en Frameworks}
Om pentests te automatiseren, zijn er verschillende frameworks en tools beschikbaar om 
efficiënt en snel informatie te verzamelen. De meeste van deze tools zijn open source, waaronder 
AutoRecon, Recon-ng, Shodan en Nmap. Deze tools zijn ontwikkeld om robust te zijn en het proces van de reconnaissance-fase te versnellen ~\autocite{Shebli}.


\subsection{Vergelijkingen}

Automatisering kan de efficiëntie en snelheid van de reconnaisance-fase verhogen. Er zijn veel tools die snel een netwerk kunnen scannen 
en de kwetsbaarheden kunnen detecteren. Door automatisatie kunnen 
organisaties sneller en kosten-effectiever tests uitvoeren. Deze automatisatie kan echter wel leiden tot een kwaliteitsverlies, 
d.w.z. dat het systeem mogelijks niet alle kwetsbaarheden vindt. ~\autocite{peris}
Meer diepgaande scans gebeuren vaak handmatig, deze zijn flexibeler dan een geautomatiseerde test. 
Met de handmatige aanpak kun je makkelijker complexe kwetsbaarheden terugvinden die niet makkelijk te detecteren vallen bij 
een geautomatiseerde tool. ~\autocite{techtarget} 
Het kiezen tussen een handmatige of een geautomatiseerde pentest komt vaak voor, deze keuze hangt af van de specifieke  
vereisten van de organisatie. In de praktijk is er vaak sprake van een hybride model in plaats van een volledige handmatig of geautomatiseerde pentest.
Dit is vaak de ideale oplossing waarbij automatisatie en menselijke ervaring elkaar kunnen 
aanvullen voor een efficiënte reconnaissance-fase. ~\autocite{techtarget}

\subsection{Efficiëntie en uitdagingen}

% Momenteel zijn er nog enkele uitdagingen bij pentests, zoals ~\textcite{Fugkeaw} beschrijft, dat er nog te veel wordt vertrouwd op de experten van het vak bij het beoordelen van de al dan niet geautomatiseerde reconnaissance-fase.
% Bij vele methoden wordt er vaak nog op gerekend dat er een menselijke input is bij het maken van een vulnerability assessment (VA) bij het analyseren en prioriteren. Dit verhoogt de kans op menselijke fouten en is tijdintensief, ook zijn de testen vaak 
% inconsistent door de individuele voorkeuren van de pentester volgens ~\textcite{Ghanem}.
% Binnen pentesting worden er nog vaak nieuwe technologieën ontdekt en gebruikt, één van deze nieuwe technologieën is het gebruik van kunstmatige intelligentie. Zoals Het Intelligent Automated Penetration Testing System (IAPTS), ontwikkeld door \textcite{Ghanem}, 
% hier wordt er gebruik gemaakt van reinforcement learning (RL) om testpatronen te leren en te optimaliseren. Als we deze technologie gebruiken in de reconnaissance-fase samen met het wiskundige model om beslissingen te maken; de Partially observed Markov decision process (POMDP) 
% kunnen we nauwkeurigere resultaten behalen. 
% Echter blijft het belangrijk dat er een juiste balans zit tussen automatisatie en menselijke controle. Menselijke testers hebben vaker meer creativiteit en intuïtie wat moeilijk is te vervangen door een geautomatiseerde tool. Daarom blijft het belangrijk om te blijven onderzoeken naar hybride modelen.

Momenteel zijn er nog enkele uitdagingen bij pentests; zoals ~\textcite{Fugkeaw} beschrijft, dat er nog te veel vertrouwen is  
in de experten van het vak bij het beoordelen van de al dan niet geautomatiseerde reconnaissance-fase. Bij veel methoden is er nog vaak de verwachting dat er een menselijke input nodig is bij het maken van een vulnerability assessment (VA) bij het analyseren en prioriteren. Dit verhoogt de kans op menselijke fouten en is tijdintensief, ook zijn de testen vaak inconsistent door de individuele voorkeuren van de pentester volgens ~\textcite{Ghanem}.
Binnen pentesting ontdekken en gebruiken professionals vaak nieuwe technologieën, één van deze nieuwe technologieën is het gebruik van kunstmatige intelligentie. Zoals het Intelligent Automated Penetration Testing System (IAPTS), dat ontwikkeld is door \textcite{Ghanem}, hier maakten zij gebruik van reinforcement learning (RL) om testpatronen te leren en te optimaliseren. Als we deze technologie gebruiken in de reconnaissance-fase samen met het wiskundige model om beslissingen te nemen, de Partially observed Markov decision proces (POMDP), kunnen we nauwkeurigere resultaten behalen. Echter blijft het belangrijk dat er een juiste balans zit tussen automatisatie en menselijke controle. Menselijke testers hebben vaker meer creativiteit en intuïtie, wat moeilijk is te vervangen door een geautomatiseerde tool. Daarom is het belangerijk onderzoek te blijven uitvoeren naar hybride modellen.  




% Hier beschrijf je de \emph{state-of-the-art} rondom je gekozen onderzoeksdomein, d.w.z.\ een inleidende, doorlopende tekst over het onderzoeksdomein van je bachelorproef. Je steunt daarbij heel sterk op de professionele \emph{vakliteratuur}, en niet zozeer op populariserende teksten voor een breed publiek. Wat is de huidige stand van zaken in dit domein, en wat zijn nog eventuele open vragen (die misschien de aanleiding waren tot je onderzoeksvraag!)?

% Je mag de titel van deze sectie ook aanpassen (literatuurstudie, stand van zaken, enz.). Zijn er al gelijkaardige onderzoeken gevoerd? Wat concluderen ze? Wat is het verschil met jouw onderzoek?

%Verwijs bij elke introductie van een term of bewering over het domein naar de vakliteratuur, bijvoorbeeld~\autocite{Hykes2013}! Denk zeker goed na welke werken je refereert en waarom.

% Draag zorg voor correcte literatuurverwijzingen! Een bronvermelding hoort thuis \emph{binnen} de zin waar je je op die bron baseert, dus niet er buiten! Maak meteen een verwijzing als je gebruik maakt van een bron. Doe dit dus \emph{niet} aan het einde van een lange paragraaf. Baseer nooit teveel aansluitende tekst op eenzelfde bron.

% Als je informatie over bronnen verzamelt in JabRef, zorg er dan voor dat alle nodige info aanwezig is om de bron terug te vinden (zoals uitvoerig besproken in de lessen Research Methods).

% Voor literatuurverwijzingen zijn er twee belangrijke commando's:
% \autocite{KEY} => (Auteur, jaartal) Gebruik dit als de naam van de auteur
%   geen onderdeel is van de zin.
% \textcite{KEY} => Auteur (jaartal)  Gebruik dit als de auteursnaam wel een
%   functie heeft in de zin (bv. ``Uit onderzoek door Doll & Hill (1954) bleek
%   ...'')

% Je mag deze sectie nog verder onderverdelen in subsecties als dit de structuur van de tekst kan verduidelijken.

%---------- Methodologie ------------------------------------------------------
\section{Methodologie}
\label{sec:methodologie}

Om de onderzoeksvraag te kunnen beantwoorden zal de aanpak een combinatie zijn van methoden, waar er onder andere gebruik
gemaak zal worden van de literatuurstudie, onderzoeken en vergelijkende analyses. Hieronder zijn de stappen beschreven:

\subsection{Literatuurstudie}

De literatuurstudie zal de basis vormen voor dit onderzoek. Door een grondige analyse te maken van de reeds bestaande tools en studies,
die bijdragen in de reconnaissance-fase van pentests, is het mogelijk om een overzicht te maken van de huidige situatie.
Hierbij spelen de vindingen van~\textcite{Shah,Kothia} een rol die de voordelen en beperkingen aantonen binnen de reconnaissance-fase.
Ook zal er rekening gehouden worden met nieuwe technologieën zoals die van ~\textcite{Ghanem,Hoang} waarin zij gebruik maken van
reinforcement learning (RL), Intelligent Automated Penetration Testing System (IAPTS) en POMDP-modellen. De implementatie van bestaande
tools en frameworks zoals AutoRecon en Recon-ng omvat ook een grondige analyse. ~\autocite{Shebli}
De resultaten van deze literatuurstudie dienen om een kader te schetsen voor de verdere ontwikkeling van dit onderzoek.

\subsection{Experimenteel onderzoek}

Een Proof of Concept (PoC) zal de effectiviteit van de automatisering in de reconnaisance-fase kunnn demonstreren.
% Om de effectiviteit van de automatisering te kunnen demonstreren in de reconnaissance-fase zal er een Proof of Concept (PoC) opgestart worden.
Dit omvat de implementatie van een workflow met behulp van automatisatie tools waaronder AutoRecon en Recon-ng. 
Deze workflow test de werking op een gesimuleerd netwerk waar er bekende kwetsbaarheden op staan om via deze data de snelheid,
nauwkeurigheid en consistentie te meten. Deze data kunnen nieuwe technologieën voortbrengen om automatisatie te optimaliseren
zoals reinforcement learning (RL) om patronen te herkennen.
De verzamelde informatie zal als basis dienen voor de evaluatie op nauwkeurigheid, tijdswinst en consistentie.


%----------------------------------------------------------------------------------------------------------
% \newpage
% \subsection{Meetbare Criteria voor Analyse}

% Om de resultaten van de analyse te beoordelen zal er gewerkt worden met de volgende meetbare criteria:
% \begin{itemize}
%     \item De nodige tijd voor informatieverzameling.
%     \item De Nauwkeurigheid van de ondekte gegevens.
%     \item Het aantal gevonden unieke kwetsbaarheden.
% \end{itemize}

% Ook zal het framework OSTTMM gebruikt worden om de validiteit van het onderzoek nog te vergroten.
% Deze richtlijnen en standaarden zal helpen om de resultaten van de PoC te vergelijken en te beoordelen.

% Door deze criteria is het makkelijker om objectief te bepalen welke aanpak het meeste effect heeft en efficiënt is.

%----------------------------------------------------------------------------------------------------------
\subsection{OSSTMM framework}

Om de validiteit van het onderzoek te vergroten zal deze studie gebruik maken van het OSSTMM (Open Source Security Testing Methodology Manual) framework. 
Dit framework biedt een gestructureerde aanpak voor het uitvoeren en maakt de resultaten van de pentest nauwkeuriger. 
De OSSTMM Manual biedt een wetenschappelijk onderbouwd kader als ideale omgeving om te plannen en tests uit te voeren in afzonderlijke, begrijpbare delen.

De OSSTMM framework stelt dat "Testing is a complicated affair and with anything complicated, you approach it in small, comprehensible pieces to be sure you don’t make mistakes." ~\autocite{Herzog}
Dit principe zal de kern vormen van de methodologie. 
Door de reconnaissance-fase op te splitsen in kleinere en duidelijke stappen, crieërt deze studie een betrouwbare manier om tools en frameworks te evalueren en toe te passen om de efficiëntie en effectiviteit te verbeteren.

\begin{itemize}
    \item Defineren van de scope: 
    de componenten die binnen de reconnaisance-fase behoren zoals netwerkstructuren, systemen en applicaties moeten duidelijk afgebakend zijn;
    \item identificeren van informatiebronnen:
    de relevante informatie voor de reconnaisance-fase zoals bronnen en informatie, openbare Databases en netwerkdiensten, moet verzameld worden; 
    \item toepassen van de OSSTMM principes:
    de OSSTMM richtlijnen en principes leggen hun nadruk op het verzamelen van objectieve en verifieerbare gegevens;
    \item evalueren van Tools en Frameworks:
    binnen de gestructureerde aanpak van OSSTMM zullen verschillende automatiseringstools en frameworks getest en beoordeld worden om hun effectiviteit en efficiëntie te bepalen;
    \item analyseren en rapporteren:
    de laatste stap van de methodologie van OSSTMM is het analyseren van de verzamelde gegevens en het opstellen van een rapport.
\end{itemize}

% wij zullen ons hier een deel 
% Door de richtlijnen en standaarden van OSSTMM te volgen, kunnen we ervoor zorgen dat de resultaten van ons onderzoek betrouwbaar en reproduceerbaar zijn.

% Het OSSTMM framework zal worden toegepast in de volgende stappen van ons onderzoek:
% \begin{itemize}
%     \item \textbf{Planning en voorbereiding}: Het definiëren van de scope en doelstellingen van de penetratietest, evenals het identificeren van de benodigde middelen en tools.
%     \item \textbf{Informatie verzamelen}: Het gebruik van OSSTMM-methoden om systematisch informatie te verzamelen over het doelwit, inclusief netwerkconfiguraties, services en mogelijke kwetsbaarheden.
%     \item \textbf{Analyse en evaluatie}: Het toepassen van OSSTMM-richtlijnen om de verzamelde informatie te analyseren en de beveiligingsstatus van het doelwit te beoordelen.
%     \item \textbf{Rapportage}: Het documenteren van de bevindingen en aanbevelingen volgens de OSSTMM-standaarden, zodat de resultaten duidelijk en bruikbaar zijn voor de betrokken stakeholders.
% \end{itemize}

% Door het OSSTMM framework te integreren in ons onderzoek, kunnen we een gestructureerde en methodische aanpak garanderen, wat bijdraagt aan de betrouwbaarheid en validiteit van onze bevindingen.

% "Testing is a complicated affair and with anything complicated, you approach it in
% small, comprehensible pieces to be sure you don’t make mistakes. " ~\autocite{Herzog}



\subsection{Vergelijkende studie}

Een vergelijkende analyse zal de toegevoegde waarde van de automatisering bepalen. binnen het onderzoek verloopt dit in 3 stappen:

\begin{itemize}
    \item een handmatig uitgevoerde reconnaissance-fase, waarbij menselijke testers traditionele methoden toepassen;
    \item een volledig geautomatiseerde aanpak met behulp van de PoC;
    \item een hybride aanpak die handmatige en geautomatiseerde methoden combineert.
\end{itemize}

De resultaten van de analyse zullen verder vergeleken worden op basis van enkele criteria zoals tijdsduur, nauwkeurigheid en gevonden kwetsbaarheden.

\newpage
\subsection{Tools en hulpmiddelen}

Dit onderzoek maakt gebruik van volgende tools: 

\begin{itemize}
    \item het verzamelen van informatie gebeurt a.d.h.v. AutoRecon, Nmap, Amass en Recon-ng;
    \item het ontwerpen van de PoC zal gebeuren met behulp van python of andere talen;
    \item de simulatie maakt gebruik van een lokaal virtueel gesimuleerd netwerk.
\end{itemize}

De machine learning libraries (bijv. TensorFlow of PyTorch) kunnen ook dienen voor de experimenten met reinforcement learning.

\subsection{Planning en deliverables}
Er zijn 4 fasen in het onderzoek:
\begin{enumerate}
    \item \textbf{Literatuurstudie} (3 weken): relevante bronnen analyseren en de onderzoeksresultaten identificeren;
    \item \textbf{Proof of Concept} (5 weken): de geautomatiseerde workflow ontwikkelen en implementeren;
    \item \textbf{Experimentele evaluatie} (4 weken): de PoC testen in een gesimuleerde omgeving;
    \item \textbf{Vergelijkende analyse en rapportage} \newline (3 weken): de resultaten vergelijken en het schrijven van het eindrapport.
\end{enumerate}


De belangrijkste deliverables die het onderzoek oogt te bereiken is het automatiseren van de reconnaissance-fase op
basis van een onderzoeksrapport met aanbevelingen en het uitrollen van een Proof of Concept om de voordelen van automatisatie te demonstreren.

% Hier beschrijf je hoe je van plan bent het onderzoek te voeren. Welke onderzoekstechniek ga je toepassen om elk van je onderzoeksvragen te beantwoorden? Gebruik je hiervoor literatuurstudie, interviews met belanghebbenden (bv.~voor requirements-analyse), experimenten, simulaties, vergelijkende studie, risico-analyse, PoC, \ldots?

% Valt je onderwerp onder één van de typische soorten bachelorproeven die besproken zijn in de lessen Research Methods (bv.\ vergelijkende studie of risico-analyse)? Zorg er dan ook voor dat we duidelijk de verschillende stappen terug vinden die we verwachten in dit soort onderzoek!

% Vermijd onderzoekstechnieken die geen objectieve, meetbare resultaten kunnen opleveren. Enquêtes, bijvoorbeeld, zijn voor een bachelorproef informatica meestal \textbf{niet geschikt}. De antwoorden zijn eerder meningen dan feiten en in de praktijk blijkt het ook bijzonder moeilijk om voldoende respondenten te vinden. Studenten die een enquête willen voeren, hebben meestal ook geen goede definitie van de populatie, waardoor ook niet kan aangetoond worden dat eventuele resultaten representatief zijn.

% Uit dit onderdeel moet duidelijk naar voor komen dat je bachelorproef ook technisch voldoen\-de diepgang zal bevatten. Het zou niet kloppen als een bachelorproef informatica ook door bv.\ een student marketing zou kunnen uitgevoerd worden.

% Je beschrijft ook al welke tools (hardware, software, diensten, \ldots) je denkt hiervoor te gebruiken of te ontwikkelen.

% Probeer ook een tijdschatting te maken. Hoe lang zal je met elke fase van je onderzoek bezig zijn en wat zijn de concrete \emph{deliverables} in elke fase?

%---------- Verwachte resultaten ----------------------------------------------
\section{Verwacht resultaat, conclusie}%
\label{sec:verwachte_resultaten}

% Hier beschrijf je welke resultaten je verwacht. Als je metingen en simulaties uitvoert, kan je hier al mock-ups maken van de grafieken samen met de verwachte conclusies. Benoem zeker al je assen en de onderdelen van de grafiek die je gaat gebruiken. Dit zorgt ervoor dat je concreet weet welk soort data je moet verzamelen en hoe je die moet meten.

% Wat heeft de doelgroep van je onderzoek aan het resultaat? Op welke manier zorgt jouw bachelorproef voor een meerwaarde?

% Hier beschrijf je wat je verwacht uit je onderzoek, met de motivatie waarom. Het is \textbf{niet} erg indien uit je onderzoek andere resultaten en conclusies vloeien dan dat je hier beschrijft: het is dan juist interessant om te onderzoeken waarom jouw hypothesen niet overeenkomen met de resultaten.

Het onderzoek zal een concreet inzicht geven in hoe de efficiëntie verbeterd kan worden met behulp van
automatisatie in de reconnaissance-fase van de pentest. De implementatie van een hybride model lijkt het efficiëntst
zal zijn waarbij het moeglijk is een geavanceerde tool en framework zoals AutoRecon, Shodan, Nmap en Recon-ng te gebruiken,
gecombineerd met menselijke expertise. Dit zal er voor zorgen dat er een snellere, betrouwbaardere en nauwkeurigere verzameling van informatie is.
Dit onderzoek zal bijdragen aan een meer efficiënte pentests, ook zal dit meer inzicht kunnen bieden in hoe organisaties tijd en kosten
kunnen besparen zonder kwaliteitsverlies.
De verwachte uitkomsten omvatten:

\begin{itemize}
    \item een gedetailleerd onderzoeksrapport van bestaande tools en frameworks om de pentests te automatiseren;
    \item een Proof of Concept uitrollen om de voordelen van een hybride model te kunnen demonstreren;
    \item organisaties krijgen concrete aanbevelingen voor het verbeteren van hun pentestomgeving door automatisering.
\end{itemize}

Deze resultaten Zullen afhankelijk zijn van testen die uitgevoerd zijn in simulaties, dit geeft een goede basis om 
pentest-workflow te verbeteren in de praktijk.